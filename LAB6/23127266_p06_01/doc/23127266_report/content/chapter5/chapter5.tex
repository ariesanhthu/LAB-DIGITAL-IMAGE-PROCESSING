\chapter{Conclusion}

\section{Task Completion Summary}

\begin{table}[H]
\centering
\begin{tabular}{p{0.15\textwidth} p{0.6\textwidth} p{0.15\textwidth}}
\hline
\textbf{No.} & \textbf{Objective} & \textbf{Completion} \\
\hline
1 & Implement a Feed Forward Neural Network with manual backpropagation in PyTorch & 100\% \\
2 & Visualize input images and network-related results & 100\% \\
3 & Convert image datasets into MNIST-like CSV format for training & 100\% \\
4 & Train and evaluate the model on both sample and full datasets & 100\% \\
5 & Visualize training loss, accuracy, and confusion matrices & 100\% \\
6 & Save and reload trained model parameters for validation & 100\% \\
7 & Compare the results with different hidden sizes and learning rates & 100\% \\
\hline
\end{tabular}
\caption{Objective completion summary}
\label{tab:objective-completion}
\end{table}


\section{Conclusion}

\section{Conclusion}

This report presented the design, implementation, and evaluation of a Feed Forward Neural Network for handwritten digit classification using an MNIST-like dataset.
The experimental results demonstrate that the proposed approach is able to achieve high classification accuracy with stable convergence behavior, especially when trained on the full training set.

\textbf{Best Practices.}
Effective dataset preprocessing through CSV-based representation simplifies data loading and ensures reproducibility.
Using a training sample for early verification and hyperparameter tuning helps reduce development time and detect implementation errors.
Training on the full dataset significantly improves generalization, while monitoring validation loss and accuracy provides a reliable indicator of overfitting.

\textbf{Limitations.}
The model shows overfitting when trained on limited data due to its fully connected architecture.
In addition, the absence of convolutional layers limits the model’s ability to exploit spatial information in images, which restricts its performance compared to more advanced architectures.

Overall, the full training set configuration offers the best balance between accuracy and generalization and is recommended for final evaluation.
