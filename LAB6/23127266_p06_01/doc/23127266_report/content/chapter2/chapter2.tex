\chapter{Network Architecture and Implementation}

\section{Overview}

The handwritten digit classification problem is formulated as a supervised multi-class classification task, where the objective is to learn a discriminative mapping from high-dimensional pixel space to discrete class labels~\cite{tds_mlp_handwritten_digits}. The input space consists of grayscale images of size $28 \times 28$ pixels, which are flattened into latent representations of dimension 784~\cite{digit_eda_varianceexplained}. Each latent vector $\mathbf{x} \in \mathbb{R}^{784}$ encodes the spatial intensity distribution of a handwritten digit, where pixel values are normalized to the range $[0, 1]$.

The classification problem aims to assign each input latent vector $\mathbf{x}_i$ to one of ten digit classes $\mathcal{C} = \{0, 1, 2, \ldots, 9\}$. Given a training dataset $\mathcal{D} = \{(\mathbf{x}_i, y_i)\}_{i=1}^{N}$ where $y_i \in \mathcal{C}$ denotes the ground-truth label, the model learns a function $f: \mathbb{R}^{784} \rightarrow \mathbb{R}^{10}$ that maps input features to a probability distribution over classes. The predicted class is obtained via the argmax operation: $\hat{y}_i = \arg\max_k f(\mathbf{x}_i)_k$.

\begin{figure}[H]
    \centering
    \includegraphics[width=0.6\textwidth]{img/theory/architecture.png}
    \caption{Feed-forward neural network architecture for digit classification.}
    \label{fig:architecture}
\end{figure}

\section{Network Construction}

The network is organized as a sequential structure using a modular design approach, where layers are dynamically added and executed in order during training and inference. This design supports flexible configuration of network architecture with multiple hidden layers and provides a clear framework for conducting implementation-based experiments.

\subsection{Class Structure}

The program is organized in an object-oriented manner with the following main classes:

\begin{itemize}
    \item \textbf{BaseLayer}: An abstract base class that defines a common interface for all layers through two methods \texttt{forward()} and \texttt{backward()}. This abstraction allows uniform treatment of different layer types in the network.
    \item \textbf{FCLayer}: A fully-connected (linear) layer that performs linear transformation between input and output, including both weights and bias. Each FCLayer stores weight matrix $W$ and bias vector $b$, initialized with small random values from a normal distribution.
    \item \textbf{ActivationLayer}: An activation layer that applies a nonlinear function element-wise and computes the corresponding gradient in the backward propagation phase. This layer wraps activation functions and their derivatives, supporting sigmoid, tanh, ReLU, and softmax.
    \item \textbf{Network}: A class that manages the entire network, responsible for adding layers, configuring the loss function, training the model, and making predictions. It maintains a list of layers and coordinates forward and backward passes.
    \item \textbf{Activation}: A utility class providing static methods for activation functions (sigmoid, tanh, ReLU, softmax).
    \item \textbf{ActivationPrime}: A utility class providing static methods for activation function derivatives used in backpropagation.
    \item \textbf{Loss}: A utility class providing static methods for loss functions (MSE, cross-entropy with logits, cross-entropy with probabilities).
    \item \textbf{LossPrime}: A utility class providing static methods for loss function derivatives.
\end{itemize}

\subsection{Network Architecture}

The implemented feed-forward neural network consists of multiple layers: an input layer, one or more hidden layers, and an output layer. The architecture transforms the input latent representation through a series of linear transformations and nonlinear activations to produce class probability estimates.

For the MNIST digit classification task, the default architecture is:
\begin{itemize}
    \item \textbf{Input layer}: Receives flattened image vectors of dimension $d_{\text{in}} = 784$, corresponding to $28 \times 28$ pixel images.
    \item \textbf{Hidden layers}: Contains configurable numbers of neurons. The default configuration uses two hidden layers with sizes $d_{h1} = 128$ and $d_{h2} = 64$, where each neuron applies a linear transformation followed by the ReLU activation function.
    \item \textbf{Output layer}: Contains $d_{\text{out}} = 10$ neurons, one for each digit class, with the Softmax activation function applied to produce normalized probability distributions.
\end{itemize}

The network architecture can be represented as: $784 \rightarrow 128 \rightarrow 64 \rightarrow 10$, where arrows indicate fully-connected layers with activation functions applied between layers.

\begin{figure}[H]
    \centering
    \includegraphics[width=0.8\textwidth]{img/a.png}
    \caption{FFNN architecture with 2 hidden layers for digit classification.}
    \label{fig:architecture}
\end{figure}

\subsection{Parameter Initialization}

The network parameters consist of weight matrices and bias vectors for each fully-connected layer. For a network with architecture $784 \rightarrow 128 \rightarrow 64 \rightarrow 10$, the parameters are:
\begin{itemize}
    \item $\mathbf{W}_1 \in \mathbb{R}^{784 \times 128}$: Weight matrix connecting input to first hidden layer
    \item $\mathbf{b}_1 \in \mathbb{R}^{128}$: Bias vector for first hidden layer
    \item $\mathbf{W}_2 \in \mathbb{R}^{128 \times 64}$: Weight matrix connecting first to second hidden layer
    \item $\mathbf{b}_2 \in \mathbb{R}^{64}$: Bias vector for second hidden layer
    \item $\mathbf{W}_3 \in \mathbb{R}^{64 \times 10}$: Weight matrix connecting second hidden layer to output
    \item $\mathbf{b}_3 \in \mathbb{R}^{10}$: Bias vector for output layer
\end{itemize}

Initialization follows a small-variance normal distribution strategy:
\begin{align}
    \mathbf{W}_i &\sim \mathcal{N}(0, \sigma^2), \quad \sigma = 0.01 \\
    \mathbf{b}_i &= \mathbf{0}
\end{align}

This initialization scheme ensures that initial activations remain in the linear region of the ReLU function, promoting stable gradient flow during early training stages.

\section{Activation Functions}

To introduce nonlinearity to the model, this practice implements four common activation functions: \textbf{sigmoid}, \textbf{tanh}, \textbf{ReLU}, and \textbf{softmax}. These functions are defined in the \texttt{Activation} class, while their corresponding derivatives are implemented in the \texttt{ActivationPrime} class for backpropagation.

\subsection{Sigmoid Function}

The sigmoid function is defined as follows:
\[
\sigma(z) = \frac{1}{1 + e^{-z}}
\]

This function maps any real value to the range $(0,1)$, commonly used in binary classification problems. In the program, sigmoid is directly implemented using PyTorch tensor operators to ensure computational efficiency.

The derivative of sigmoid with respect to its input is:
\[
\sigma'(z) = \sigma(z)(1 - \sigma(z))
\]

\subsection{Tanh Function}

The hyperbolic tangent (tanh) function is defined by:
\[
\tanh(z) = \frac{e^z - e^{-z}}{e^z + e^{-z}}
\]

Unlike sigmoid, the tanh function has a range of $(-1,1)$ and is zero-centered, which helps stabilize gradient propagation in many cases.

The derivative of tanh is:
\[
\tanh'(z) = 1 - \tanh^2(z)
\]

\subsection{ReLU Activation Function}

\begin{figure}[H]
    \centering
    \includegraphics[width=0.5\textwidth]{img/theory/ReLU.png}
    \caption{ReLU activation function and its derivative. \cite{gfg_relu_activation}}
    \label{fig:relu}
\end{figure}

The Rectified Linear Unit (ReLU) activation function is applied element-wise to the hidden layer pre-activations. The function is defined as:

\begin{equation}
\text{ReLU}(z) = \max(0, z) = \begin{cases}
z & \text{if } z > 0 \\
0 & \text{if } z \leq 0
\end{cases}
\end{equation}

The derivative of ReLU with respect to its input is:

\begin{equation}
\frac{d}{dz}\text{ReLU}(z) = \begin{cases}
1 & \text{if } z > 0 \\
0 & \text{if } z \leq 0
\end{cases}
\end{equation}

\subsubsection{Rationale for ReLU}

ReLU is chosen for the hidden layer activation due to several advantageous properties:
\begin{itemize}
    \item \textbf{Computational efficiency}: The function and its derivative are computationally inexpensive, involving only thresholding operations.
    \item \textbf{Sparsity}: ReLU naturally induces sparsity by zeroing out negative activations, effectively reducing the effective network capacity and promoting feature selectivity.
    \item \textbf{Gradient flow}: Unlike saturating activations such as sigmoid or tanh, ReLU avoids vanishing gradients for positive inputs, allowing deeper networks to train effectively. The gradient remains constant (equal to 1) for positive values, facilitating stable backpropagation.
    \item \textbf{Nonlinearity}: Despite its piecewise linear nature, the combination of multiple ReLU units enables the network to approximate complex nonlinear decision boundaries.
\end{itemize}

\subsection{Softmax Activation Function}

\begin{figure}[H]
    \centering
    \includegraphics[width=0.5\textwidth]{img/theory/softmax-function.png}
    \caption{Softmax activation function mapping logits to probability distribution.}
    \label{fig:softmax}
\end{figure}

The Softmax function is applied to the output layer pre-activations to produce a valid probability distribution over the ten digit classes. Given a vector $\mathbf{z} \in \mathbb{R}^{10}$ of logits, the Softmax function computes:

\begin{equation}
\text{Softmax}(\mathbf{z})_k = \frac{\exp(z_k - \max_j z_j)}{\sum_{j=1}^{10} \exp(z_j - \max_j z_j)}
\end{equation}

where the subtraction of the maximum value ($\max_j z_j$) is performed for numerical stability, preventing overflow in the exponential computation.

The Softmax output satisfies the probability axioms:

\begin{align}
\sum_{k=1}^{10} \text{Softmax}(\mathbf{z})_k &= 1 \\
\text{Softmax}(\mathbf{z})_k &\geq 0 \quad \forall k \in \{1, \ldots, 10\}
\end{align}

\begin{figure}[H]
    \centering
    \includegraphics[width=0.5\textwidth]{img/theory/softmax-visual.PNG}
    \caption{Softmax example visualization. \cite{sefiks_softmax_activation}}
    \label{fig:softmax-visual}
\end{figure}

\subsubsection{Rationale for Softmax}

\begin{figure}[H]
    \centering
    \includegraphics[width=0.5\textwidth]{img/theory/visualize with probility.PNG}
    \caption{Softmax architecture in multi-class classification. \cite{tds_mlp_handwritten_digits}}
    \label{fig:softmax-architecture}
\end{figure}

Softmax is the appropriate choice for the output layer in multi-class classification problems for the following reasons:
\begin{itemize}
    \item \textbf{Probability interpretation}: The output directly represents class probabilities, enabling intuitive interpretation of model confidence. The argmax operation on Softmax outputs yields the most probable class prediction.
    \item \textbf{Compatibility with cross-entropy loss}: Softmax pairs naturally with the cross-entropy loss function, resulting in a simplified gradient computation during backpropagation. The gradient of the cross-entropy loss with respect to the logits reduces to the difference between predicted probabilities and one-hot encoded targets.
    \item \textbf{Competitive normalization}: The exponential function in Softmax creates a competitive normalization effect, where larger logit values receive exponentially higher probabilities, effectively amplifying differences between classes.
    \item \textbf{Differentiability}: Softmax is smooth and differentiable everywhere, ensuring stable gradient-based optimization.
\end{itemize}

\subsection{Gradient Computation for Softmax}

The partial derivative of Softmax with respect to its input logits is required for backpropagation. For a Softmax output $\mathbf{p} = \text{Softmax}(\mathbf{z})$, when receiving gradient $\frac{\partial L}{\partial \mathbf{p}}$ from the loss function, the gradient with respect to logits is computed as:

\begin{equation}
\frac{\partial L}{\partial z_k} = \sum_j \frac{\partial L}{\partial p_j} \frac{\partial p_j}{\partial z_k} = p_k \left(\frac{\partial L}{\partial p_k} - \sum_j \frac{\partial L}{\partial p_j} p_j\right)
\end{equation}

This formulation is implemented in the \texttt{ActivationPrime.softmax\_derivative()} method, which takes both the pre-activation logits and the output error gradient as arguments.

\section{Forward Propagation}

Forward propagation computes the network output by sequentially applying linear transformations and activation functions to the input latent representation. The process transforms the input through each layer, storing intermediate values for use in backpropagation.

\subsection{Forward Pass Computation}

Given an input batch $\mathbf{X} \in \mathbb{R}^{N \times 784}$ containing $N$ samples, the forward propagation proceeds through the following steps:

\subsubsection{Input to Hidden Layer}

The first linear transformation computes pre-activations for the hidden layer:
\begin{equation}
\mathbf{Z}_1 = \mathbf{X}\mathbf{W}_1 + \mathbf{b}_1
\end{equation}

where $\mathbf{Z}_1 \in \mathbb{R}^{N \times d_h}$ contains the pre-activation values. The ReLU activation is then applied element-wise:
\begin{equation}
\mathbf{H}_1 = \text{ReLU}(\mathbf{Z}_1) = \max(\mathbf{0}, \mathbf{Z}_1)
\end{equation}

yielding the hidden layer activations $\mathbf{H}_1 \in \mathbb{R}^{N \times d_h}$.

\subsubsection{Hidden to Output Layer}

The hidden activations are transformed through the second linear layer:
\begin{equation}
\mathbf{Z}_2 = \mathbf{H}_1\mathbf{W}_2 + \mathbf{b}_2
\end{equation}

where $\mathbf{Z}_2 \in \mathbb{R}^{N \times 10}$ contains the output logits. The Softmax function is applied row-wise to produce the final probability distribution:
\begin{equation}
\hat{\mathbf{Y}} = \text{Softmax}(\mathbf{Z}_2)
\end{equation}

where $\hat{\mathbf{Y}} \in \mathbb{R}^{N \times 10}$ and each row sums to unity.

\subsection{Example: Forward Pass with Small Matrices}

Consider a simplified example with $N=2$ samples, $d_{\text{in}}=4$ (reduced input dimension), and $d_h=3$ hidden neurons. The forward pass computation proceeds as follows:

\subsubsection{Step 1: Input to Hidden}

Given input matrix and parameters:
\begin{align}
\mathbf{X} &= \begin{bmatrix}
0.2 & 0.5 & 0.8 & 0.3 \\
0.1 & 0.9 & 0.4 & 0.6
\end{bmatrix}, \quad
\mathbf{W}_1 = \begin{bmatrix}
0.1 & 0.2 & -0.1 \\
0.3 & -0.2 & 0.4 \\
-0.1 & 0.3 & 0.2 \\
0.2 & 0.1 & -0.3
\end{bmatrix}, \quad
\mathbf{b}_1 = \begin{bmatrix}
0.1 & -0.1 & 0.2
\end{bmatrix}
\end{align}

The pre-activation computation yields:
\begin{equation}
\mathbf{Z}_1 = \mathbf{X}\mathbf{W}_1 + \mathbf{b}_1 = \begin{bmatrix}
0.2 & 0.5 & 0.8 & 0.3 \\
0.1 & 0.9 & 0.4 & 0.6
\end{bmatrix} \begin{bmatrix}
0.1 & 0.2 & -0.1 \\
0.3 & -0.2 & 0.4 \\
-0.1 & 0.3 & 0.2 \\
0.2 & 0.1 & -0.3
\end{bmatrix} + \begin{bmatrix}
0.1 & -0.1 & 0.2
\end{bmatrix}
\end{equation}

\begin{equation}
\mathbf{Z}_1 = \begin{bmatrix}
0.25 & 0.35 & 0.15 \\
0.42 & 0.18 & 0.28
\end{bmatrix}
\end{equation}

Applying ReLU activation:
\begin{equation}
\mathbf{H}_1 = \text{ReLU}(\mathbf{Z}_1) = \begin{bmatrix}
0.25 & 0.35 & 0.15 \\
0.42 & 0.18 & 0.28
\end{bmatrix}
\end{equation}

\subsubsection{Step 2: Hidden to Output}

Given output layer parameters:
\begin{align}
\mathbf{W}_2 = \begin{bmatrix}
0.2 & -0.1 & 0.3 & 0.1 & -0.2 & 0.15 & 0.05 & -0.1 & 0.25 & 0.0 \\
-0.1 & 0.3 & -0.2 & 0.2 & 0.1 & -0.15 & 0.3 & 0.05 & -0.1 & 0.2 \\
0.1 & -0.2 & 0.15 & -0.1 & 0.3 & 0.2 & -0.15 & 0.25 & 0.1 & -0.05
\end{bmatrix}, \quad \\
\mathbf{b}_2 = \begin{bmatrix}
0.05 & -0.05 & 0.1 & 0.0 & 0.05 & -0.1 & 0.15 & -0.05 & 0.1 & 0.0
\end{bmatrix}
\end{align}

Note: In practice, $\mathbf{W}_2$ has shape $d_h \times 10$, but for brevity, this example uses a reduced output dimension. The computation proceeds:
\begin{equation}
\mathbf{Z}_2 = \mathbf{H}_1\mathbf{W}_2 + \mathbf{b}_2
\end{equation}

After computing logits, Softmax is applied row-wise. For a sample logit vector $\mathbf{z}_2^{(i)} = [z_1, z_2, \ldots, z_{10}]$, the Softmax output is:
\begin{equation}
p_k = \frac{\exp(z_k - \max_j z_j)}{\sum_{j=1}^{10} \exp(z_j - \max_j z_j)}
\end{equation}

yielding the final probability distribution $\hat{\mathbf{Y}}$ where each row represents class probabilities for one sample.

\section{Backward Propagation}

Backward propagation is used to compute the gradients of the loss function with respect to all trainable parameters in the network and to update these parameters in order to minimize the loss.

The network in this practice is trained using the cross-entropy loss function (with softmax probabilities), defined as:
\[
L = -\frac{1}{N} \sum_{i=1}^{N} \log(\hat{y}_{i,c_i}),
\]
where $c_i \in \{0,1,\dots,9\}$ denotes the ground-truth integer class label for sample $i$, $\hat{y}_{i,c_i}$ is the predicted probability for the true class, and $N$ is the batch size.

\subsection{Gradient Computation}

The backpropagation process is implemented manually and follows the reverse order of the forward pass. For each training batch, gradients are propagated from the output layer back to the input layer using the chain rule.

\paragraph{Gradient from the loss function.}
The derivative of the cross-entropy loss with respect to the predicted probabilities is given by:
\[
\frac{\partial L}{\partial \hat{y}_{i,k}} = \begin{cases}
-\frac{1}{N \cdot \hat{y}_{i,c_i}} & \text{if } k = c_i \\
0 & \text{otherwise}
\end{cases}
\]

However, when using softmax with cross-entropy, it is more efficient to compute the gradient directly with respect to the logits. The gradient with respect to logits $\mathbf{z}_3$ is:
\[
\frac{\partial L}{\partial \mathbf{z}_3} = \frac{1}{N}(\hat{\mathbf{Y}} - \mathbf{Y}_{\text{one-hot}})
\]

where $\mathbf{Y}_{\text{one-hot}}$ is the one-hot encoded version of the integer labels. This elegant form arises from the combination of cross-entropy loss and Softmax activation.

\paragraph{Backpropagation through the softmax activation layer.}
The softmax derivative computation is handled specially in the \texttt{ActivationLayer.backward()} method. When the activation is softmax, the derivative function receives both the pre-activation logits and the output error gradient:

\[
\frac{\partial L}{\partial \mathbf{z}_3} = \text{softmax\_derivative}(\mathbf{z}_3, \frac{\partial L}{\partial \hat{\mathbf{Y}}})
\]

The softmax derivative computes:
\[
\frac{\partial L}{\partial z_{3,k}} = p_k \left(\frac{\partial L}{\partial \hat{y}_k} - \sum_j \frac{\partial L}{\partial \hat{y}_j} p_j\right)
\]

where $p_k = \text{Softmax}(\mathbf{z}_3)_k$.

\paragraph{Backpropagation through the activation layer (ReLU).}
Each ReLU activation layer applies a non-linear function element-wise. During backpropagation, the gradient with respect to the pre-activation input $z$ is computed by:
\[
\frac{\partial L}{\partial z} = \frac{\partial L}{\partial a} \odot \text{ReLU}'(z),
\]
where $a = \text{ReLU}(z)$ is the activation output, $\odot$ denotes element-wise multiplication, and $\text{ReLU}'(z) = 1$ if $z > 0$, else $0$.

The activation layer does not update any parameters; it only propagates the gradient backward.

\paragraph{Backpropagation through the fully-connected layer.}
A fully-connected layer performs the linear transformation:
\[
Z = XW + b,
\]
where $X$ is the input vector, $W$ is the weight matrix, and $b$ is the bias. Given the gradient with respect to the output $Z$, $\delta = \partial L / \partial Z$, the gradients are computed as:
\[
\frac{\partial L}{\partial W} = X^\top \delta,
\]
\[
\frac{\partial L}{\partial b} = \sum \delta \quad \text{(sum over batch dimension)},
\]
\[
\frac{\partial L}{\partial X} = \delta W^\top.
\]

\paragraph{Parameter update using mini-batch SGD.}

The network parameters are updated using Stochastic Gradient Descent (SGD) with mini-batches. With a learning rate $\alpha$, the update rules are:

\[
W \leftarrow W - \alpha \frac{\partial L}{\partial W},
\]
\[
b \leftarrow b - \alpha \frac{\partial L}{\partial b}.
\]

Since training is performed in mini-batches, the gradients are averaged over the batch size $N$, and the bias gradient is computed as a sum over the batch dimension.

\paragraph{Backward propagation order.}

During training, the backward pass iterates through all layers in reverse order compared to the forward pass. At each layer, gradients are computed and propagated to the preceding layer until gradients with respect to the network input are obtained. The \texttt{Network.fit()} method coordinates this process by calling \texttt{backward()} on each layer in reverse order.

\subsection{Implementation Details}

The \texttt{backward()} method of each layer receives the output error gradient and learning rate as arguments. For \texttt{FCLayer}, it computes parameter gradients, updates weights and biases, and returns the input error gradient. For \texttt{ActivationLayer}, it computes the gradient through the activation function and returns the pre-activation error gradient.

The special handling for softmax derivative requires passing both the pre-activation logits and the output error gradient, which is implemented in the \texttt{ActivationPrime.softmax\_derivative()} method.

\section{Training Process}

\subsection{Input Data}

The training dataset used in this practice is the MNIST-like handwritten digit dataset. Images are stored in folder structure, where each subdirectory is labeled by digit class (0-9). Each image is converted to grayscale, resized to $28 \times 28$ pixels, and flattened into a 784-dimensional vector.

The input--output pairs are defined as:
\[
\mathcal{D} = \{(x_i, y_i)\}_{i=1}^{N},
\]
where $x_i \in \mathbb{R}^{784}$ represents a flattened and normalized image, and $y_i \in \{0,1,\dots,9\}$ denotes the ground-truth digit label.

In the implementation, the input data is stored as a tensor of shape $(N, 784)$, and the target labels are stored as a tensor of shape $(N,)$ with integer class labels. All values are represented using floating-point tensors for inputs and integer tensors for labels.

\subsection{Data Preprocessing}

\subsubsection{Image Loading and Preprocessing}

Raw grayscale images are loaded from folder structure and preprocessed through the following steps:
\begin{enumerate}
    \item \textbf{Image loading}: Images are loaded from subdirectories organized by digit class.
    \item \textbf{Grayscale conversion}: Images are converted to grayscale if they are in color format.
    \item \textbf{Resizing}: Images are resized to $28 \times 28$ pixels to match MNIST format.
    \item \textbf{Flattening}: Each $28 \times 28$ image is flattened into a 784-dimensional vector.
    \item \textbf{Normalization}: Pixel values are normalized using standardization:
    \[
    x_{i,j}^{\text{std}} = \frac{x_{i,j} - \mu_j}{\sigma_j}
    \]
    where $\mu_j$ and $\sigma_j$ are the mean and standard deviation of pixel $j$ computed over the training set.
\end{enumerate}

\subsubsection{Train-Validation Split}

The dataset is split into training and validation sets. A fixed 20\% of the samples is randomly selected as the validation set, while the remaining data is used for training. The split is controlled by a fixed random seed to ensure reproducibility across experiments.

\subsection{Training Procedure}

Training is performed using mini-batch Stochastic Gradient Descent (SGD). During training, the dataset is divided into mini-batches of configurable size (default: 64 samples per batch).

For each epoch, the network iterates through all training batches and performs the following steps for each batch:

\begin{quote}
\begin{enumerate}
    \item Forward propagation through all layers to compute the predicted probability distributions.
    \item Loss computation using the cross-entropy loss function.
    \item Backward propagation through all layers to compute gradients.
    \item Parameter update using SGD with the specified learning rate.
\end{enumerate}
\end{quote}

After all batches have been processed in an epoch, the model is evaluated on both training and validation sets. Loss values and accuracy metrics are computed and stored in the training history.

\subsection{Training Configuration}

The training process is controlled by several hyperparameters:
\begin{itemize}
    \item \textbf{Learning rate} ($\alpha$): Controls the step size in parameter updates (default: 0.05).
    \item \textbf{Batch size}: Number of samples processed together in each update (default: 64).
    \item \textbf{Number of epochs}: Total number of complete passes through the training dataset (default: 200 for sample set, 100 for full set).
    \item \textbf{Hidden layer sizes}: Configurable architecture, default $[128, 64]$.
    \item \textbf{Activation function}: ReLU for hidden layers, Softmax for output layer.
    \item \textbf{Normalization mode}: Standardization is applied to input features.
\end{itemize}

\subsection{Validation and Monitoring}

After each epoch (or at specified intervals), the model performance is evaluated on the validation set. Key metrics tracked include:
\begin{itemize}
    \item \textbf{Training loss}: Cross-entropy loss on training set.
    \item \textbf{Validation loss}: Cross-entropy loss on validation set.
    \item \textbf{Training accuracy}: Percentage of correctly classified samples in training set.
    \item \textbf{Validation accuracy}: Percentage of correctly classified samples in validation set.
\end{itemize}

These metrics are plotted over epochs to visualize training progress and detect overfitting. A large gap between training and validation accuracy may indicate overfitting, while consistent improvement in both metrics indicates effective learning.

\subsection{Training History}

The training history is stored in a dictionary containing lists of metrics at each evaluation point:
\begin{itemize}
    \item \texttt{epochs}: List of epoch numbers where metrics were recorded.
    \item \texttt{train\_loss}: List of training loss values.
    \item \texttt{val\_loss}: List of validation loss values.
    \item \texttt{train\_acc}: List of training accuracy values.
    \item \texttt{val\_acc}: List of validation accuracy values.
\end{itemize}

Metrics are recorded at the first epoch, last epoch, and at regular intervals (e.g., every 10 epochs) to reduce storage requirements while maintaining visibility into training progress.

\section{Model Persistence}

Model persistence enables saving trained network parameters to disk and reloading them for inference or continued training. This capability is essential for deploying models and reproducing experimental results.

\subsection{State Dictionary Representation}

The model state is represented as a dictionary containing all learnable parameters from fully-connected layers. The \texttt{Network} class provides two methods for managing state:

\begin{itemize}
    \item \texttt{state\_dict()}: Returns a dictionary containing weights and biases for all fully connected layers. Keys follow the format \texttt{layer\_\{i\}\_weights} and \texttt{layer\_\{i\}\_bias} where \texttt{i} is the index of the FCLayer in the network (counting only FC layers, not activation layers).
    \item \texttt{load\_state\_dict(state\_dict)}: Loads weights and biases from a state dictionary into the network, with validation to ensure the structure matches. Raises \texttt{ValueError} if the number of layers or keys don't match.
\end{itemize}

For a network with architecture $784 \rightarrow 128 \rightarrow 64 \rightarrow 10$, the state dictionary includes:
\begin{itemize}
    \item \texttt{layer\_0\_weights}: $\mathbf{W}_1 \in \mathbb{R}^{784 \times 128}$ (input-to-first-hidden weight matrix)
    \item \texttt{layer\_0\_bias}: $\mathbf{b}_1 \in \mathbb{R}^{128}$ (first hidden layer bias vector)
    \item \texttt{layer\_1\_weights}: $\mathbf{W}_2 \in \mathbb{R}^{128 \times 64}$ (first-to-second-hidden weight matrix)
    \item \texttt{layer\_1\_bias}: $\mathbf{b}_2 \in \mathbb{R}^{64}$ (second hidden layer bias vector)
    \item \texttt{layer\_2\_weights}: $\mathbf{W}_3 \in \mathbb{R}^{64 \times 10}$ (second-hidden-to-output weight matrix)
    \item \texttt{layer\_2\_bias}: $\mathbf{b}_3 \in \mathbb{R}^{10}$ (output layer bias vector)
\end{itemize}

At initialization, weights are randomly sampled from a normal distribution. When a saved \texttt{state\_dict} is loaded, these initial random values are completely replaced by the trained weights, ensuring consistent model restoration.

\subsection{Saving Model}

The model is saved using the \texttt{save(filepath)} method of the \texttt{Network} class. This method:

\begin{enumerate}
    \item Extracts the state dictionary via \texttt{self.state\_dict()}
    \item Extracts the network architecture information from FCLayers (list of layer sizes: $[input\_size, hidden1, hidden2, \ldots, output\_size]$)
    \item Combines architecture and state dictionary into a single dictionary:
    \begin{itemize}
        \item \texttt{architecture}: Network structure information (e.g., $[784, 128, 64, 10]$)
        \item \texttt{state\_dict}: Trained weights and biases
    \end{itemize}
    \item Serializes the dictionary to disk using Python's \texttt{pickle} module
    \item Saves the file with the specified path (e.g., \texttt{custom\_mnist\_net.pkl})
\end{enumerate}

The implementation uses \texttt{pickle.dump()} to write the combined dictionary to a binary file. This approach stores both the model parameters and architecture information, enabling automatic network reconstruction during loading.

\subsection{Loading Model}

To load a saved model, the static method \texttt{Network.load(filepath, network=None)} is used:

\begin{enumerate}
    \item The saved data is loaded from the file using \texttt{pickle.load()}
    \item If the saved format includes \texttt{architecture} (new format), a new network is automatically created:
    \begin{itemize}
        \item Network architecture is reconstructed from the saved architecture list
        \item FCLayers and ActivationLayers are added in the correct sequence
        \item Default activation (tanh) and loss function (MSE) are configured
    \end{itemize}
    \item If the saved format only contains \texttt{state\_dict} (old format), a network instance must be provided with matching architecture
    \item The state dictionary is applied to the network instance via \texttt{network.load\_state\_dict()}
    \item The method validates that the state dictionary structure matches the network architecture
\end{enumerate}

The \texttt{load\_state\_dict()} method performs validation to ensure:
\begin{itemize}
    \item The number of items in the state dictionary matches the number of fully connected layers (2 items per layer: weights and bias)
    \item All required keys (\texttt{layer\_\{i\}\_weights} and \texttt{layer\_\{i\}\_bias}) are present for each layer
    \item Raises \texttt{ValueError} if there is a mismatch
\end{itemize}

After loading, the network produces identical predictions to those obtained before saving, given the same input data and architecture configuration.

\subsection{Checkpoint Saving and Loading}

For training workflows, a checkpoint system is implemented that saves both model parameters and configuration:

\begin{itemize}
    \item \texttt{save\_checkpoint(path, network, cfg)}: Saves network state dictionary together with configuration dictionary to a checkpoint file.
    \item \texttt{load\_checkpoint(path)}: Loads checkpoint and automatically rebuilds the network using the saved configuration, returning both the network and configuration.
\end{itemize}

This approach enables complete model restoration including hyperparameters and training configuration, facilitating experiment reproducibility and model deployment.

\subsection{Usage Example}

The typical workflow for saving and loading a model is:

\begin{quote}
\begin{enumerate}
    \item \textbf{Train the model}: Create and train a network with desired architecture and hyperparameters using \texttt{Network.fit()}
    \item \textbf{Save the model}: Call \texttt{net.save("custom\_mnist\_net.pkl")} to persist the trained weights and architecture
    \item \textbf{Load the model}: Call \texttt{Network.load("custom\_mnist\_net.pkl")} to automatically reconstruct and restore the network
    \item \textbf{Use for inference}: Call \texttt{net.predict(X)} to make predictions on new data
\end{enumerate}
\end{quote}

Alternatively, for checkpoint-based workflows:
\begin{quote}
\begin{enumerate}
    \item \textbf{Save checkpoint}: Call \texttt{save\_checkpoint("checkpoint.pkl", net, config)} after training
    \item \textbf{Load checkpoint}: Call \texttt{loaded = load\_checkpoint("checkpoint.pkl")} to get network and config
    \item \textbf{Verify}: Test the loaded model on validation data to confirm identical predictions
\end{enumerate}
\end{quote}

After loading, the model can be used for inference through forward propagation. Given the same input data and network architecture, the loaded model is expected to produce the same predictions as those obtained before saving.

