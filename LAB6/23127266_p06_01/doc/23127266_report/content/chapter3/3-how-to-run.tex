\subsection{Running the Notebook}

The implementation is provided as a Jupyter Notebook (\texttt{23127266\_p06\_01.ipynb}). Execute cells sequentially by clicking "Run" or pressing Shift+Enter.

\subsection{Execution Workflow}

\begin{enumerate}
    \item \textbf{Environment Setup}: Import required libraries and verify installation
    \item \textbf{Dataset Download}: Acquire dataset using Method 1 or Method 2
    \item \textbf{Preprocessing}: Convert images to CSV format using \texttt{build\_mnist\_like\_csv()}
    \item \textbf{Model Definition}: Define \texttt{FFNeuralNetwork} class with ReLU and Softmax activations
    \item \textbf{Data Loading}: Load and normalize data using \texttt{load\_mnist\_csv()}
    \item \textbf{Training}: Train model using \texttt{train\_ffnn\_from\_csv()}
    \item \textbf{Evaluation}: Visualize results and compute accuracy metrics
    \item \textbf{Model Persistence}: Save trained models to disk
\end{enumerate}

\subsection{Model Files}

After training, the following model files are saved:

\begin{itemize}
    \item \texttt{ffnn\_mnist\_state\_dict.pt}: Model trained on training sample
    \item \texttt{ffnn\_mnist\_state\_dict\_set.pt}: Model trained on full training set
\end{itemize}

These files contain the model's state dictionary (weights and biases) serialized using PyTorch's \texttt{torch.save()} function.

