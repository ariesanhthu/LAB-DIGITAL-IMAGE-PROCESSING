\chapter{Installation and Usage}

This chapter describes the implementation of a custom feed-forward neural network for MNIST digit classification using PyTorch. The implementation includes activation functions, loss functions, layer architectures, and a complete training pipeline. This section covers environment setup, dataset preparation, implementation structure, and usage instructions.

\section{Environment Setup}

\subsection{System Requirements}

This edge detection project requires the following system components:

\begin{itemize}
    \item Python 3.7+;
    \item CUDA (optional, for GPU-based training) \\
    For training:
    \item RAM: Minimum 8GB;
    \item Disk space: Approximately 5GB for dataset and checkpoints.
\end{itemize}

\subsection{Python and Library Installation}

To set up the development environment, perform the following steps:

\textbf{a) Create a Virtual Environment.}

Create a virtual environment to isolate the project's dependencies.

\begin{quote}
\begin{lstlisting}[style=pseudo]
python -m venv venv
\end{lstlisting}
\end{quote}

Activate the virtual environment:

\begin{quote}
\begin{lstlisting}[style=pseudo]
# Windows:
venv\Scripts\activate

# Linux/Mac:
source venv/bin/activate
\end{lstlisting}
\end{quote}

\textbf{b) Install PyTorch.}

Depending on your system, choose one of the following commands:

\begin{quote}
\begin{lstlisting}[style=pseudo]
# CPU only:
pip install torch torchvision torchaudio --index-url https://download.pytorch.org/whl/cpu

# CUDA 11.8:
pip install torch torchvision torchaudio --index-url https://download.pytorch.org/whl/cu118

# CUDA 12.1:
pip install torch torchvision torchaudio --index-url https://download.pytorch.org/whl/cu121
\end{lstlisting}
\end{quote}

\textbf{c) Install Other Libraries.}

Install the remaining dependencies:

\begin{quote}
\begin{lstlisting}[style=pseudo]
pip install numpy pillow opencv-python tqdm matplotlib scikit-image scipy
\end{lstlisting}
\end{quote}

Alternatively, use the requirements.txt file:
\begin{quote}
\begin{lstlisting}[style=pseudo]
pip install -r requirements.txt
\end{lstlisting}
\end{quote}

\subsection{Main Libraries}

The project uses the following Python libraries:

\begin{itemize}
    \item \texttt{numpy}: Multidimensional array processing and mathematical operations;
    \item \texttt{opencv-python}: Image processing and manipulation;
    \item \texttt{Pillow}: Reading and writing image files;
    \item \texttt{torch}, \texttt{torchvision}: PyTorch deep learning framework;
    \item \texttt{matplotlib}: Plotting and visualizing images and figures;
    \item \texttt{scikit-image}: Advanced image processing algorithms;
    \item \texttt{scipy}: Scientific and numerical computation library;
    \item \texttt{tqdm}: Progress bar display.
\end{itemize}

\subsection{Dataset Setup}

The project uses the BIPED dataset. The dataset should be organized as follows:

\begin{quote}
\begin{verbatim}
dataset/
`-- BIPED/
    `-- edges/
        |-- imgs/
        |   |-- train/
        |   |   `-- rgbr/
        |   |       `-- real/
        |   `-- test/
        |       `-- rgbr/
        |-- edge_maps/
        |   |-- train/
        |   |   `-- rgbr/
        |   |       `-- real/
        |   `-- test/
        |       `-- rgbr/
        |-- train_rgb.lst
        `-- test_rgb.lst
\end{verbatim}
\end{quote}

\subsection{Installation Check}

To verify that the installation was successful, run the main script:

\begin{quote}
\begin{lstlisting}[style=pseudo]
cd source
python main.py
\end{lstlisting}
\end{quote}

\section{Dataset Download}

The MNIST dataset in JPG format is required for training and evaluation. The dataset is available at \url{https://www.kaggle.com/datasets/scolianni/mnistasjpg}.

\subsection{Method 1: Direct Download from Kaggle}

\begin{enumerate}
    \item Navigate to the dataset page on Kaggle
    \item Click the "Download" button to download the dataset archive
    \item Extract the archive to the project directory
    \item Rename the extracted folder to \texttt{mnistasjpg\_data}
\end{enumerate}

\begin{figure}[H]
    \centering
    \includegraphics[width=0.8\textwidth]{img/chapter3/Kaggle.PNG}
    \caption{Kaggle dataset download page for MNIST as JPG.}
    \label{fig:kaggle-download}
\end{figure}

\subsection{Method 2: Programmatic Download via Kaggle API}

Alternatively, uncomment and execute the following code cells in the notebook:

\textbf{Step 1: Install kagglehub library}

\begin{quote}
\begin{lstlisting}[style=pseudo]
# %pip install kagglehub
\end{lstlisting}
\end{quote}

\textbf{Step 2: Download dataset}

\begin{quote}
\begin{lstlisting}[style=pseudo]
# import kagglehub
# path = kagglehub.dataset_download("scolianni/mnistasjpg")
# print("Path to dataset files:", path)
\end{lstlisting}
\end{quote}

\textbf{Step 3: Move dataset to target directory}

\begin{quote}
\begin{lstlisting}[style=pseudo]
# import shutil
# import os
# target_path = os.path.join(os.getcwd(), "mnistasjpg_data")
# if not os.path.exists(target_path):
#     shutil.move(path, target_path)
#     print(f"Dataset moved to: {target_path}")
\end{lstlisting}
\end{quote}

\subsection{Dataset Structure}

After download, the dataset directory \texttt{mnistasjpg\_data} contains the following structure:

\begin{itemize}
    \item \texttt{trainingSample/}: Reduced training set (600 samples)
    \item \texttt{trainingSet/}: Full training set (42,000 samples)
\end{itemize}

Each split folder contains subdirectories labeled 0-9, where each subdirectory contains JPG images of the corresponding digit class.



\section{Implementation Structure}

This section provides an overview of the core components implemented in the neural network framework.

\subsection{Core Classes}

The implementation consists of several key classes organized into functional modules.

\subsubsection{Activation Functions}

The \texttt{Activation} class provides static methods for non-linear activation functions used in neural network layers.

\begin{table}[H]
\centering
\caption{Activation Class Methods}
\begin{tabular}{|l|p{8cm}|p{3cm}|}
\hline
\textbf{Method} & \textbf{Description} & \textbf{Output} \\
\hline
\texttt{relu(x)} & Computes Rectified Linear Unit activation: $\max(0, x)$ & \texttt{torch.Tensor} \\
\hline
\texttt{softmax(x)} & Computes stable softmax activation for multi-class classification. Input must be 2D tensor of shape (N, C) & \texttt{torch.Tensor} \\
\hline
\end{tabular}
\end{table}

\subsubsection{Activation Derivatives}

The \texttt{ActivationPrime} class provides derivatives of activation functions required for backpropagation.

\begin{table}[H]
\centering
\caption{ActivationPrime Class Methods}
\begin{tabular}{|l|p{5cm}|p{3cm}|}
\hline
\textbf{Method} & \textbf{Description} & \textbf{Output} \\
\hline
\texttt{relu\_derivative(z)} & Computes derivative of ReLU w.r.t. pre-activation z & \texttt{torch.Tensor} \\
\hline
\texttt{softmax\_derivative(z, out\_error)} & Computes gradient through softmax layer. Requires pre-activation z and output error gradient & \texttt{torch.Tensor} \\
\hline
\end{tabular}
\end{table}

\subsubsection{Loss Functions}

The \texttt{Loss} class implements loss functions for training neural networks.

\begin{table}[H]
\centering
\caption{Loss Class Methods}
\begin{tabular}{|p{6cm}|p{4cm}|p{3cm}|p{3cm}|}
\hline
\textbf{Method} & \textbf{Description} & \textbf{Input} & \textbf{Output} \\
\hline
\texttt{cross\_entropy\_with\_logits (y\_true\_int, logits)} & Computes cross-entropy loss from raw logits using stable formulation & \texttt{y\_true\_int: torch.Tensor (N,), logits: torch.Tensor (N, C)} & \texttt{torch.Tensor} (scalar) \\
\hline
\texttt{cross\_entropy (y\_true\_int, probs)} & Computes cross-entropy loss from probability tensor & \texttt{y\_true\_int: torch.Tensor (N,), probs: torch.Tensor (N, C)} & \texttt{torch.Tensor} (scalar) \\
\hline
\end{tabular}
\end{table}

\subsubsection{Loss Derivatives}

The \texttt{LossPrime} class provides derivatives of loss functions for gradient computation.

\begin{table}[H]
\centering
\caption{LossPrime Class Methods}
\begin{tabular}{|p{5cm}|p{3cm}|p{4cm}|p{3cm}|}
\hline
\textbf{Method} & \textbf{Description} & \textbf{Input} & \textbf{Output} \\
\hline
\texttt{cross\_entropy\_with\_logits\_prime (y\_true\_int, logits)} & Computes derivative of cross-entropy w.r.t. logits & \texttt{y\_true\_int: torch.Tensor (N,), logits: torch.Tensor (N, C)} & \texttt{torch.Tensor} (N, C) \\
\hline
\texttt{cross\_entropy\_prime (y\_true\_int, probs)} & Computes derivative of cross-entropy w.r.t. probabilities & \texttt{y\_true\_int: torch.Tensor (N,), probs: torch.Tensor (N, C)} & \texttt{torch.Tensor} (N, C) \\
\hline
\end{tabular}
\end{table}

\subsubsection{Base Layer}

The \texttt{BaseLayer} abstract class defines the interface for all neural network layers.

\begin{table}[H]
\centering
\caption{BaseLayer Class Properties and Methods}
\begin{tabular}{|l|p{9cm}|}
\hline
\textbf{Property/Method} & \textbf{Description} \\
\hline
\texttt{forward(in\_data)} & Abstract method for forward propagation. Must be implemented by subclasses \\
\hline
\texttt{backward(out\_error, rate)} & Abstract method for backward propagation and parameter updates. Must be implemented by subclasses \\
\hline
\end{tabular}
\end{table}

\subsubsection{Fully-Connected Layer}

The \texttt{FCLayer} class implements a fully-connected (linear) layer with weights and bias.

\begin{table}[H]
\centering
\caption{FCLayer Class Properties and Methods}
\begin{tabular}{|l|p{9cm}|}
\hline
\textbf{Property/Method} & \textbf{Description} \\
\hline
\texttt{weights} & Weight matrix of shape (in\_size, out\_size), initialized with small random values \\
\hline
\texttt{bias} & Bias vector of shape (1, out\_size), initialized to zeros \\
\hline
\texttt{in\_data} & Cached input tensor from forward pass \\
\hline
\texttt{out\_data} & Cached output tensor from forward pass \\
\hline
\texttt{\_\_init\_\_(in\_size, out\_size, init\_std)} & Initializes layer with input/output dimensions and weight initialization standard deviation \\
\hline
\texttt{forward(in\_data)} & Computes linear transformation: $out = in \times W + b$ \\
\hline
\texttt{backward(out\_error, rate)} & Performs backpropagation, computes gradients, and updates weights/bias using SGD \\
\hline
\end{tabular}
\end{table}

\subsubsection{Activation Layer}

The \texttt{ActivationLayer} class applies non-linear activation functions element-wise.

\begin{table}[H]
\centering
\caption{ActivationLayer Class Properties and Methods}
\begin{tabular}{|l|p{5cm}|}
\hline
\textbf{Property/Method} & \textbf{Description} \\
\hline
\texttt{activation} & Activation function callable (e.g., sigmoid, tanh, relu, softmax) \\
\hline
\texttt{activation\_derivative} & Derivative function callable for backpropagation \\
\hline
\texttt{in\_data} & Cached pre-activation tensor from forward pass \\
\hline
\texttt{out\_data} & Cached activated tensor from forward pass \\
\hline
\texttt{\_\_init\_\_(activation, activation\_derivative)} & Initializes layer with activation function and its derivative \\
\hline
\texttt{forward(in\_data)} & Applies activation function to input tensor \\
\hline
\texttt{backward(out\_error, rate)} & Computes gradient through activation function (rate parameter unused) \\
\hline
\end{tabular}
\end{table}

\subsubsection{Network Class}

The \texttt{Network} class is a sequential container for neural network layers with training and inference capabilities.

\begin{table}[H]
\centering
\caption{Network Class Properties and Methods}
\begin{tabular}{|p{7cm}|p{7cm}|}
\hline
\textbf{Property/Method} & \textbf{Description} \\
\hline
\texttt{layers} & List of layer objects in sequential order \\
\hline
\texttt{loss} & Loss function callable \\
\hline
\texttt{loss\_prime} & Loss function derivative callable \\
\hline
\texttt{\_\_init\_\_()} & Initializes empty network with no layers \\
\hline
\texttt{add(layer)} & Adds a layer to the network \\
\hline
\texttt{use(loss, loss\_prime)} & Sets loss function and its derivative for training \\
\hline
\texttt{predict(data)} & Forward propagates a single sample through all layers \\
\hline
\texttt{predicts(data)} & Forward propagates multiple samples, returns list of outputs \\
\hline
\texttt{fit(x\_train, y\_train, epochs, alpha, print\_every)} & Trains network using sample-wise SGD with specified epochs and learning rate \\
\hline
\texttt{state\_dict()} & Returns dictionary containing all trainable parameters (weights and biases) \\
\hline
\texttt{load\_state\_dict(state\_dict)} & Loads weights and biases from state dictionary \\
\hline
\texttt{save(filepath)} & Saves network state (architecture and weights) to file using pickle \\
\hline
\texttt{load(filepath, network)} & Static method that loads network from file and automatically reconstructs architecture \\
\hline
\end{tabular}
\end{table}

\subsection{Utility Functions}

The implementation includes utility functions for data loading, network building, training, and evaluation.

\begin{table}[H]
\centering
\caption{Data Loading and Network Building Functions}
\begin{tabular}{|p{6cm}|p{3cm}|p{3cm}|p{3cm}|}
\hline
\textbf{Function} & \textbf{Description} & \textbf{Input} & \textbf{Output} \\
\hline
\texttt{load\_mnist\_from \_image\_folders(cfg)} & Loads MNIST images from folder structure, performs train/val split, normalization, and returns tensors & \texttt{cfg: dict} with dataset config & \texttt{dict} with X\_train, y\_train, X\_val, y\_val, mean, std, paths \\
\hline
\texttt{build\_network\_mnist(cfg)} & Builds MNIST classifier network with configurable hidden layers and activations & \texttt{cfg: dict} with model config & \texttt{Network} instance \\
\hline
\texttt{get\_activation \_functions(name)} & Maps activation name to (activation, derivative) tuple & \texttt{name: str} ("relu", "tanh", "sigmoid", "softmax") & \texttt{tuple} of callables \\
\hline
\end{tabular}
\end{table}

\begin{longtable}{|p{6cm}|p{3cm}|p{4cm}|p{4cm}|}
\caption{Training and Evaluation Functions} \label{tab:training-eval-functions} \\
\hline
\textbf{Function} & \textbf{Description} & \textbf{Input} & \textbf{Output} \\
\hline
\endfirsthead

\multicolumn{4}{c}%
{{\bfseries \tablename\ \thetable{} -- continued from previous page}} \\
\hline
\textbf{Function} & \textbf{Description} & \textbf{Input} & \textbf{Output} \\
\hline
\endhead

\hline \multicolumn{4}{|r|}{{Continued on next page}} \\ \hline
\endfoot

\hline
\endlastfoot

\texttt{train\_digit\_classifier(cfg)} & Complete training pipeline for digit classification using mini-batch SGD & \texttt{cfg: dict} with training config & \texttt{dict} with network, history, data \\
\hline
\texttt{train\_and\_evaluate(cfg, checkpoint\_path, single\_image\_path)} & Complete pipeline: load data, train, evaluate, visualize, and save checkpoint & \texttt{cfg: dict}, optional checkpoint/image paths & \texttt{dict} with network, history, data, results \\
\hline
\texttt{accuracy\_from\_probs(probs, y\_true\_int)} & Computes classification accuracy from probability tensor and integer labels & \texttt{probs: torch.Tensor (N, C), y\_true\_int: torch.Tensor (N,)} & \texttt{float} accuracy \\
\hline
\texttt{plot\_history(history)} & Plots training curves (loss and accuracy) over epochs & \texttt{history: dict} with training metrics & \texttt{None} (displays plots) \\
\hline
\texttt{confusion\_matrix\_np(y\_true, y\_pred, num\_classes)} & Computes confusion matrix for classification evaluation & \texttt{y\_true, y\_pred: np.ndarray (N,), num\_classes: int} & \texttt{np.ndarray} (C, C) \\
\hline
\texttt{show\_confusion\_matrix(cm, title)} & Visualizes confusion matrix as heatmap & \texttt{cm: np.ndarray (C, C), title: str} & \texttt{None} (displays plot) \\
\hline
\texttt{describe\_network(net)} & Extracts and prints FC-layer architecture & \texttt{net: Network} & \texttt{list[int]} architecture \\
\hline
\texttt{visualize\_network\_structure(net, title)} & Visualizes network architecture as flow diagram & \texttt{net: Network, title: str} & \texttt{None} (displays plot) \\
\hline
\texttt{visualize\_samples(paths, y\_true, num\_show, title)} & Displays grid of sample images with labels & \texttt{paths: list[str], y\_true: np.ndarray, num\_show: int, title: str} & \texttt{None} (displays plot) \\
\hline
\texttt{predict\_single\_image(path, net, mean, std, device)} & Predicts digit class for a single image file & \texttt{path: str, net: Network, mean/std: np.ndarray, device: str} & \texttt{tuple[int, np.ndarray]} (pred, probs) \\
\hline
\texttt{visualize\_predictions(net, paths, y\_true, mean, std, device, num\_show)} & Visualizes predictions on validation images with true/predicted labels & \texttt{net: Network, paths: list[str], y\_true: np.ndarray, mean/std: np.ndarray, device: str, num\_show: int} & \texttt{None} (displays plot) \\
\hline
\texttt{save\_checkpoint(path, network, cfg)} & Saves network weights and config to file & \texttt{path: str, network: Network, cfg: dict} & \texttt{None} \\
\hline
\texttt{load\_checkpoint(path)} & Loads checkpoint and rebuilds network & \texttt{path: str} & \texttt{dict} with network and config \\
\hline
\end{longtable}

\section{Implementation Structure and Execution Steps}

% \subsection{Implementation Sections}

\begin{longtable}{|p{1cm}|p{3.5cm}|p{8.5cm}|}
\hline
\textbf{No.} & \textbf{Section Name} & \textbf{Summary (Inputs / Outputs / Properties)} \\
\hline
\endfirsthead
\hline
\textbf{No.} & \textbf{Section Name} & \textbf{Summary (Inputs / Outputs / Properties)} \\
\hline
\endhead
\hline
\endfoot
\hline
\caption{Implementation sections in the notebook}
\label{tab:implementation-sections}
\endlastfoot

1 & Import and Setup &
Inputs: none.  
Outputs: initialized environment, fixed random seed. \\
\hline

2 & Activation &
Class properties: activation function, derivative function.  
Inputs: tensor $x$.  
Outputs: activated tensor and gradient tensor. \\
\hline

3 & Loss &
Class properties: loss function, loss derivative.  
Inputs: predicted output $y_{\text{pred}}$, target $y$.  
Outputs: scalar loss value and loss gradient. \\
\hline

4 & Layers &
\texttt{FCLayer} properties: weights, bias, input data, output data.  
\texttt{ActivationLayer} properties: activation function, activation derivative.  
Inputs: input tensor.  
Outputs: transformed tensor and propagated gradient. \\
\hline

5 & Network &
Class properties: list of layers, loss function, learning rate.  
Inputs: training data and target values.  
Outputs: trained network and prediction results. \\
\hline

6 & Training Data &
Inputs: XOR input samples and corresponding labels.  
Outputs: formatted tensors for training and evaluation. \\
\hline

7 & Builder Utilities &
Inputs: layer sizes, activation types, input data.  
Outputs: constructed network instances, accuracy values, visualization figures. \\
\hline

8 & Training and Testing &
Inputs: network instance, hyperparameters.  
Outputs: accuracy metrics and decision boundary visualizations. \\
\hline

9 & Save and Load Model &
Inputs: trained network instance, file path.  
Outputs: saved model file (\texttt{xor\_model.pkl}) and loaded network instance. \\
\hline

\end{longtable}


\section{Training Configuration Parameters}

This section lists the key configuration parameters used for training the neural network on different datasets.

\subsection{Training Configuration}

The full training set configuration uses the complete dataset for final model evaluation:

\begin{itemize}
    \item \textbf{Input size}: 784
    \item \textbf{Hidden layers}: [128, 64]
    \item \textbf{Output size}: 10 classes
    \item \textbf{Epochs}: 100
    \item \textbf{Batch size}: 32
    \item \textbf{Learning rate}: 0.01
    \item \textbf{Validation split}: 0.2
    \item \textbf{Weight initialization}: Standard deviation 0.01
    \item \textbf{Normalization mode}: Standardize
    \item \textbf{Hidden activation}: ReLU
    \item \textbf{Seed}: 42
\end{itemize}