\subsection{System Requirements}

This edge detection project requires the following system components:

\begin{itemize}
    \item Python 3.7+;
    \item CUDA (optional, for GPU-based training) \\
    For training:
    \item RAM: Minimum 8GB;
    \item Disk space: Approximately 5GB for dataset and checkpoints.
\end{itemize}

\subsection{Python and Library Installation}

To set up the development environment, perform the following steps:

\textbf{a) Create a Virtual Environment.}

Create a virtual environment to isolate the project's dependencies.

\begin{quote}
\begin{lstlisting}[style=pseudo]
python -m venv venv
\end{lstlisting}
\end{quote}

Activate the virtual environment:

\begin{quote}
\begin{lstlisting}[style=pseudo]
# Windows:
venv\Scripts\activate

# Linux/Mac:
source venv/bin/activate
\end{lstlisting}
\end{quote}

\textbf{b) Install PyTorch.}

Depending on your system, choose one of the following commands:

\begin{quote}
\begin{lstlisting}[style=pseudo]
# CPU only:
pip install torch torchvision torchaudio --index-url https://download.pytorch.org/whl/cpu

# CUDA 11.8:
pip install torch torchvision torchaudio --index-url https://download.pytorch.org/whl/cu118

# CUDA 12.1:
pip install torch torchvision torchaudio --index-url https://download.pytorch.org/whl/cu121
\end{lstlisting}
\end{quote}

\textbf{c) Install Other Libraries.}

Install the remaining dependencies:

\begin{quote}
\begin{lstlisting}[style=pseudo]
pip install numpy pillow opencv-python tqdm matplotlib scikit-image scipy
\end{lstlisting}
\end{quote}

Alternatively, use the requirements.txt file:
\begin{quote}
\begin{lstlisting}[style=pseudo]
pip install -r requirements.txt
\end{lstlisting}
\end{quote}

\subsection{Main Libraries}

The project uses the following Python libraries:

\begin{itemize}
    \item \texttt{numpy}: Multidimensional array processing and mathematical operations;
    \item \texttt{opencv-python}: Image processing and manipulation;
    \item \texttt{Pillow}: Reading and writing image files;
    \item \texttt{torch}, \texttt{torchvision}: PyTorch deep learning framework;
    \item \texttt{matplotlib}: Plotting and visualizing images and figures;
    \item \texttt{scikit-image}: Advanced image processing algorithms;
    \item \texttt{scipy}: Scientific and numerical computation library;
    \item \texttt{tqdm}: Progress bar display.
\end{itemize}

\subsection{Dataset Setup}

The project uses the BIPED dataset. The dataset should be organized as follows:

\begin{quote}
\begin{verbatim}
dataset/
`-- BIPED/
    `-- edges/
        |-- imgs/
        |   |-- train/
        |   |   `-- rgbr/
        |   |       `-- real/
        |   `-- test/
        |       `-- rgbr/
        |-- edge_maps/
        |   |-- train/
        |   |   `-- rgbr/
        |   |       `-- real/
        |   `-- test/
        |       `-- rgbr/
        |-- train_rgb.lst
        `-- test_rgb.lst
\end{verbatim}
\end{quote}

\subsection{Installation Check}

To verify that the installation was successful, run the main script:

\begin{quote}
\begin{lstlisting}[style=pseudo]
cd source
python main.py
\end{lstlisting}
\end{quote}