\section{Overview}

In this assignment, I implement several edge detection techniques using NumPy. 
Color images $f(x,y)\in\mathbb{R}^3$ are first converted to grayscale $I(x,y)$, and all 
algorithms operate on this single-channel representation.

Traditional operators are reimplemented from their mathematical formulations and applied 
through discrete convolution on $I(x,y)$. A basic neural network model is also constructed 
to perform learning-based edge detection.

Finally, the results from the NumPy implementations and the neural network are compared 
with PyTorch’s built-in functions to observe differences between classical methods, 
data-driven models, and optimized library routines.

\subsection{Definition}

\textbf{What is an edge?}

An edge in a digital image is a set of pixels where the image intensity changes abruptly. These pixels mark locations of significant intensity variation, often caused by sudden changes in illumination, surface orientation, or object structure. \cite[p.~81]{gonzalezwoods2018}

\textbf{How to detect an edge?}

Edges are formed from pixels with derivative values that exceed a preset threshold. Thus, an edge is a 'local' concept that is based on a measure of intensity-level discontinuity at a point. \cite[p.~81]{gonzalezwoods2018}

\subsection{Problem Statement}

\textbf{Input.}  
A grayscale digital image
\[
f(x,y) : \Omega \subset \mathbb{Z}^2 \rightarrow [0,255],
\]
which maps each pixel coordinate $(x,y)$ to an intensity value. 
\textit{(This project uses grayscale images for processing.)}

\textbf{Goal.}  
Identify locations where $f(x,y)$ exhibits significant intensity discontinuities.

\textbf{Output.}  
An edge map
\[
E(x,y) : \Omega \rightarrow \{0,1\},
\]
defined as
\[
E(x,y) =
\begin{cases}
1, & \text{if } g(x,y) > T, \\
0, & \text{otherwise},
\end{cases}
\]
where
\[
g(x,y) = \mathcal{D}[f(x,y)]
\]
is the edge response produced by an edge operator $\mathcal{D}$.

\textbf{Problem.}  
Design or select an operator $\mathcal{D}$ such that the resulting map $E(x,y)$
accurately marks true intensity discontinuities, achieves good spatial localization,
and remains robust to noise.

\subsection{Approaches}

In this project, I implement the two main approaches:
\begin{quote}
    \begin{itemize}
        \item Traditional methods
        \item Deep learning-based methods
    \end{itemize}
\end{quote}
    
Traditional methods are the classical edge detection methods that are based on mathematical operators. Deep learning-based methods are the modern edge detection methods that are based on deep learning models. Each approach has several techniques as the figure \ref{fig:edge-detection-approaches} below.

\begin{figure}[H]
    \centering
    \resizebox{\textwidth}{!}{
    \begin{forest}
    for tree={
        draw,
        rounded corners,
        node options={align=center},
        edge={-latex},
        s sep=5pt,
        l sep=10pt,
        font=\small
    }
    [Edge Detection Approaches, s sep=20pt, l sep=20pt
        [Traditional Methods, s sep=10pt, l sep=60pt
            [Gradient Operators, s sep=2pt, l sep=60pt
                [Basic Gradient]
                [Differencing Operators]
                [Roberts]
                [Prewitt]
                [Sobel]
                [Frei-Chen]
            ]
            [Laplacian Operators
                [4-neighborhood\\Laplacian]
                [8-neighborhood\\Laplacian]
                [Laplacian mask variants]
            ]
            [Laplacian of Gaussian (LoG)]
            [Canny Edge Detector]
        ]
        [Deep Learning-based Methods
            [U-Net]
            [HED]
        ]
    ]
    \end{forest}
    }
    \caption{Overview of Edge Detection Approaches}
    \label{fig:edge-detection-approaches}
\end{figure}
    